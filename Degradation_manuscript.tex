\documentclass[]{article}
\usepackage{lmodern}
\usepackage{amssymb,amsmath}
\usepackage{ifxetex,ifluatex}
\usepackage{fixltx2e} % provides \textsubscript
\ifnum 0\ifxetex 1\fi\ifluatex 1\fi=0 % if pdftex
  \usepackage[T1]{fontenc}
  \usepackage[utf8]{inputenc}
\else % if luatex or xelatex
  \ifxetex
    \usepackage{mathspec}
  \else
    \usepackage{fontspec}
  \fi
  \defaultfontfeatures{Ligatures=TeX,Scale=MatchLowercase}
\fi
% use upquote if available, for straight quotes in verbatim environments
\IfFileExists{upquote.sty}{\usepackage{upquote}}{}
% use microtype if available
\IfFileExists{microtype.sty}{%
\usepackage{microtype}
\UseMicrotypeSet[protrusion]{basicmath} % disable protrusion for tt fonts
}{}
\usepackage[margin=1in]{geometry}
\usepackage{hyperref}
\hypersetup{unicode=true,
            pdftitle={Degradation},
            pdfauthor={David Hemprich-Bennett},
            pdfborder={0 0 0},
            breaklinks=true}
\urlstyle{same}  % don't use monospace font for urls
\usepackage{graphicx,grffile}
\makeatletter
\def\maxwidth{\ifdim\Gin@nat@width>\linewidth\linewidth\else\Gin@nat@width\fi}
\def\maxheight{\ifdim\Gin@nat@height>\textheight\textheight\else\Gin@nat@height\fi}
\makeatother
% Scale images if necessary, so that they will not overflow the page
% margins by default, and it is still possible to overwrite the defaults
% using explicit options in \includegraphics[width, height, ...]{}
\setkeys{Gin}{width=\maxwidth,height=\maxheight,keepaspectratio}
\IfFileExists{parskip.sty}{%
\usepackage{parskip}
}{% else
\setlength{\parindent}{0pt}
\setlength{\parskip}{6pt plus 2pt minus 1pt}
}
\setlength{\emergencystretch}{3em}  % prevent overfull lines
\providecommand{\tightlist}{%
  \setlength{\itemsep}{0pt}\setlength{\parskip}{0pt}}
\setcounter{secnumdepth}{0}
% Redefines (sub)paragraphs to behave more like sections
\ifx\paragraph\undefined\else
\let\oldparagraph\paragraph
\renewcommand{\paragraph}[1]{\oldparagraph{#1}\mbox{}}
\fi
\ifx\subparagraph\undefined\else
\let\oldsubparagraph\subparagraph
\renewcommand{\subparagraph}[1]{\oldsubparagraph{#1}\mbox{}}
\fi

%%% Use protect on footnotes to avoid problems with footnotes in titles
\let\rmarkdownfootnote\footnote%
\def\footnote{\protect\rmarkdownfootnote}

%%% Change title format to be more compact
\usepackage{titling}

% Create subtitle command for use in maketitle
\newcommand{\subtitle}[1]{
  \posttitle{
    \begin{center}\large#1\end{center}
    }
}

\setlength{\droptitle}{-2em}
  \title{Degradation}
  \pretitle{\vspace{\droptitle}\centering\huge}
  \posttitle{\par}
  \author{David Hemprich-Bennett}
  \preauthor{\centering\large\emph}
  \postauthor{\par}
  \predate{\centering\large\emph}
  \postdate{\par}
  \date{21 June, 2018}

\usepackage{booktabs}
\usepackage{longtable}
\usepackage{array}
\usepackage{multirow}
\usepackage[table]{xcolor}
\usepackage{wrapfig}
\usepackage{float}
\usepackage{colortbl}
\usepackage{pdflscape}
\usepackage{tabu}
\usepackage{threeparttable}
\usepackage{threeparttablex}
\usepackage[normalem]{ulem}
\usepackage{makecell}

\begin{document}
\maketitle

\section{Abstract}\label{abstract}

\section{Introduction}\label{introduction}

Habitat loss and degradation are the most common threat to the
persistence of terrestrial species (WWF, 2016), affecting large areas of
palaeotropical forests. This is particularly true of Borneo, a
biodiversity hotspot in the Malay Archipelago where 62\% of remaining
forest has been classified as being degraded (Bryan et al., 2013).
Although primary forests have been shown to be `irreplacable' in terms
of biodiversity, logged forests are more biodiverse than secondary
forest and so are worthy of conservation concern (Gibson et al., 2011).
Despite the rate of global deforestation having reduced in the last 25
years (FAO Forestry, 2015), attention must be given to degraded forest
due to the low priority given to their protection against agricultural
expansion (Gibbs et al., 2010).

\section{Causes of degradation}\label{causes-of-degradation}

\begin{itemize}
\tightlist
\item
  A natural process, being accelerated in the anthropocene
\item
  Define it. Acknowledge that there's bloody loads of definitions
\item
  Borneo-specific stuff
\item
  Habitat loss and degradation combined threaten 44.8 percent of species
  (WWF Living planet report
\item
  (from bats in the anthropocene) A pan-tropical meta-analysis of
  land-use change studies points to the irreplaceable value of
  old-growth forests, but also highlights the high species diversity
  found in regenerating logged forests compared to secondary forests
  (Gibson et al. 2011)*
\end{itemize}

\section{Consequences of degradation}\label{consequences-of-degradation}

\begin{itemize}
\tightlist
\item
  Need to consider the habitat effects for both secondary consumers,
  primary consumers and primary producers
\item
  Loss of roost sites
\item
  Work on birds in Brazil's Atlantic rainforest (unpublished) showed
  that the loss of specialist insectivorous birds led to an increase in
  insect herbivores \%(see note on Cristina Banks-Leite from the
  Ceesymposium)
\item
  Matrix type as well as distance for fragmentation is important
\item
  General trend towards functional homogenisation

  \begin{itemize}
  \tightlist
  \item
    Lower diversity of bats captured at SAFE than other two sites (need
    to use iNEXT to show this)
  \item
    Insect diversity literature for Borneo?
  \end{itemize}
\item
  What we know from the bats of the neotropics
\item
  Bat diet *We need to be careful when making comparisons, as Zhan and
  MacIsaac 2015 - Rare biosphere exploration using high-throughput
  sequencing: research progress and perspectives - state, much variation
  is caused by the random encounter of rare sequences when sampling or
  in the lab. This can cause erroneous findings of beta diversity
\end{itemize}

*Protected areas generally have higher biodiversity than non-protected
areas {[}@Gray2016{]}

\section{Methods}\label{methods}

\begin{table}[H]
\centering
\begin{tabular}{l|r|r|r|r}
\hline
  & Danum & Maliau & SAFE & SBE\\
\hline
Hice & 186.000000 & 97.000000 & 112.000000 & 57\\
\hline
Hidi & 2.000000 & 10.000000 & 3.000000 & 0\\
\hline
Hidy & 0.000000 & 13.000000 & 9.000000 & 0\\
\hline
Hiri & 3.000000 & 1.000000 & 14.000000 & 0\\
\hline
Keha & 3.000000 & 0.000000 & 23.000000 & 0\\
\hline
Kein & 29.000000 & 11.000000 & 46.000000 & 0\\
\hline
Kepa & 21.000000 & 1.000000 & 6.000000 & 0\\
\hline
Rhbo & 1.000000 & 28.000000 & 10.000000 & 0\\
\hline
Rhse & 10.000000 & 4.000000 & 14.000000 & 0\\
\hline
Rhtr & 14.000000 & 19.000000 & 29.000000 & 0\\
\hline
Shannon diversity & 1.128072 & 1.512298 & 1.804932 & 0\\
\hline
\end{tabular}
\end{table}

\section{Results}\label{results}

\subsection{Sample size effects}\label{sample-size-effects}

in Figure \ref{fig:inext} we see the different rates of MOTU
accumulation at each site.

\begin{figure}
\centering
\includegraphics{Degradation_manuscript_files/figure-latex/inext_plot-1.pdf}
\caption{\label{fig:inext}interpolation and extrapolation of MOTU
diversity at each site}
\end{figure}

\begin{table}[H]
\centering
\begin{tabular}{l|l|r|r|r|r|r|r}
\hline
  & Site & Observed number of MOTU & Estimated total number of MOTU & Estimated percent completeness & Number of samples & Estimated number of samples required & Standard error of total estimate\\
\hline
10 & all & 3535 & 6217.107 & 56.85924 & 736 & 1294.4245 & 166.351\\
\hline
7 & Danum & 1938 & 3953.179 & 49.02384 & 269 & 548.7127 & 164.658\\
\hline
4 & Maliau & 1227 & 2798.523 & 43.84456 & 185 & 421.9452 & 163.771\\
\hline
1 & SAFE & 1524 & 3979.952 & 38.29192 & 282 & 736.4478 & 225.160\\
\hline
\end{tabular}
\end{table}

\begin{figure}
\centering
\includegraphics{Degradation_manuscript_files/figure-latex/species_removal-1.pdf}
\caption{\label{fig:sp_removal}The importance of each bat species to
their network}
\end{figure}

As shown in Figure \ref{fig:sp_removal}, when removing most species
there is little effect on the networks, however the effect of removing
Hipposideros cervinus can be substantial, for example the rank order of
the networks for the metric `functional complementarity' is totally
reversed, with Danum going from being the most complementary network
when the full sample sizes are used, to the least complementary when
\emph{H. cervinus} is removed.

\subsection{Beta-diversity}\label{beta-diversity}

\begin{figure}
\centering
\includegraphics{plots/beta/betaplot.jpg}
\caption{captionabc}
\end{figure}

\subsection{Networks and reliability of
MOTU}\label{networks-and-reliability-of-motu}

\begin{figure}
\centering
\includegraphics{plots/MOTU_sites_combined.jpg}
\caption{captionabc}
\end{figure}

\begin{figure}
\centering
\includegraphics{plots/sitewise_lineplot.jpg}
\caption{captionabc}
\end{figure}

\subsection{Taxa}\label{taxa}

\begin{figure}
\centering
\includegraphics{plots/sitewise_proportion_of_bats_containing.pdf}
\caption{The proportion of bats containing each prey order obtained}
\end{figure}

\includegraphics{} ```

\section{Discussion}\label{discussion}

\subsection{The issues of comparing greatly incomplete
networks}\label{the-issues-of-comparing-greatly-incomplete-networks}

Danum showed far higher levels of functional complementarity than Maliau
and SAFE when compared at their fully-sampled extent, but this seems to
have been entirely driven by \emph{Hipposideros cervinus}, as when they
are removed then the rank order switches, with Danum having the lowest
complementarity.


\end{document}
